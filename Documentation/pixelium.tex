\documentclass[12pt, a4paper]{report}
\usepackage[magyar]{babel}
\usepackage[T1]{fontenc}
\usepackage[utf8]{inputenc}
\usepackage{graphicx}
\usepackage{listings}
\usepackage{xcolor}
\usepackage{hyperref}
\usepackage{geometry}
\usepackage{titlesec}
\usepackage{fancyhdr}
\usepackage{float}
\usepackage{amsmath}

\geometry{a4paper, margin=2.5cm}

% Színek a kódblokkokhoz
\definecolor{codegreen}{rgb}{0,0.6,0}
\definecolor{codegray}{rgb}{0.5,0.5,0.5}
\definecolor{codepurple}{rgb}{0.58,0,0.82}
\definecolor{backcolour}{rgb}{0.95,0.95,0.92}

\lstdefinestyle{mystyle}{
    backgroundcolor=\color{backcolour},   
    commentstyle=\color{codegreen},
    keywordstyle=\color{magenta},
    numberstyle=\tiny\color{codegray},
    stringstyle=\color{codepurple},
    basicstyle=\ttfamily\footnotesize,
    breakatwhitespace=false,         
    breaklines=true,                 
    captionpos=b,                    
    keepspaces=true,                 
    numbers=left,                    
    numbersep=5pt,                  
    showspaces=false,                
    showstringspaces=false,
    showtabs=false,                  
    tabsize=2
}

\lstset{style=mystyle}

% Cím formázás
\titleformat{\chapter}[display]
{\normalfont\huge\bfseries}{\chaptertitlename\ \thechapter}{20pt}{\Huge}

\begin{document}

\begin{titlepage}
    \centering
    \vspace*{2cm}
    {\Huge\bfseries Pixelium Képszerkesztő\par}
    \vspace{1cm}
    {\LARGE Dokumentáció\par}
    \vspace{2cm}
    \includegraphics[width=0.4\textwidth]{icon.jpeg}\par
    \vfill
    {\large \today\par}
\end{titlepage}

\tableofcontents

\chapter{Bevezetés}
\section{A Projekt Célja}
A Pixelium egy nagy teljesítményű, többplatformos képszerkesztő alkalmazás, amely a .NET 9.0 és Avalonia UI technológiákra épül. Célja, hogy alapvető képszerkesztési funkciókat biztosítson gyors és modern felületen, a párhuzamosság és gyorsítótárak kihasználásával.

\section{Főbb Jellemzők}
\begin{itemize}
    \item \textbf{Nagy teljesítmény}: Párhuzamos feldolgozés és memória-optimalizálás
    \item \textbf{Multiplatform}: Linux, Windows, macOS támogatás
    \item \textbf{Réteges szerkesztés}: Rétegkezelésen alapuló szerkesztés
    \item \textbf{Szűrők}: Gauss elmosás, élérzékelés, sarokdetektálás
    \item \textbf{Hisztogram analízis}: Hisztogram generálás a bemeneti képről
    \item \textbf{Visszavonás/ismétlés}: Undo/Redo művelet
\end{itemize}

\chapter{Telepítés és Futtatás}
\section{Előfeltételek}
A Pixelium Fedora 42 Linux rendszeren került fejlesztésre, szem előtt tartva a keretrendszerek kiválasztásakor a multiplatform támogatást. A program futtatható Linux, Windows és Mac rendszereken is.

\section{Fordítás a Forráskódból}

A Pixelium fordításához a .NET 9.0 SDK-ra van szükség.

\subsection{Gyors fordítás}
\begin{lstlisting}[language=bash]
dotnet build Pixelium.sln

dotnet run --project Pixelium.UI/Pixelium.UI.csproj
\end{lstlisting}

\subsection{Többplatformos fordítás}
A \texttt{build-all.sh} szkript segítségével minden támogatott platformra elkészíthetők a bináris fájlok:

\begin{lstlisting}[language=bash]
./build-all.sh
\end{lstlisting}

A fordítási eredmények a \texttt{builds/} könyvtárban találhatók gzip állományba csomagolva.

\section{Támogatott Platformok}
\begin{itemize}
    \item \textbf{Linux}: x64, ARM64 architektúrák
    \item \textbf{Windows}: 10/11 (x64, ARM64), Önálló egyetlen EXE fájl vagy EXE és DLL-ek
    \item \textbf{macOS}: Intel és Apple Silicon processzorok, (x64, ARM64)
\end{itemize}

\section{Az Alkalmazás Indítása}

\subsection{Forráskódból történő indítás}

Az alkalmazás forráskódból a következő módon indítható el:

\begin{enumerate}
    \item Terminálablak megnyitása
    \item Navigáció a projekt könyvtárába
    \item Parancs futtatása: \texttt{dotnet run --project Pixelium.UI}
\end{enumerate}

\subsection{Előre buildelt binárisok indítása}

A kész, buildelt binárisok esetében az alkalmazás elindul a forráskód vagy telepítés nélkül.

\chapter{Felhasználói Útmutató}
\section{Kezdő lépések}
\subsection{Alkalmazás indítása}
A program indítása után egy beépített mintakép jelenik meg, ahol lehet tesztelni a szűrőket. A későbbiekben másik beépített minta kép is választható, vagy a tallózással megnyitható egy meglévő kép.

\subsection{Kép megnyitása}
A kép megnyitásához több lehetőség is rendelkezésre áll:
\begin{itemize}
    \item A \textbf{Fájl > Megnyitás} menüpont kiválasztása
    \item A \texttt{Ctrl+O} billentyűkombináció használata
    \item A támogatott képformátumok közül választás (PNG, JPEG, BMP)
\end{itemize}

\subsection{Mintaképek betöltése}
Az alkalmazás beépített mintaképeket tartalmaz, amelyek megkönnyítik a különböző funkciók tesztelését:
\begin{itemize}
    \item \textbf{Minta 1: Gradiens és szöveg} - Általános szűrőtesztelésre
    \item \textbf{Minta 2: Geometriai formák} - Él- és sarokdetektálás vizsgálatára
    \item \textbf{Minta 3: Színteszt mintázat} - Hisztogram és gamma korrekció tesztelésére
\end{itemize}

\section{A Felület Ismertetése}
Az alkalmazás felülete több fő területre oszlik:
\begin{itemize}
    \item \textbf{Menüsor}: Az alapvető műveletek és beállítások elérésére szolgál
    \item \textbf{Rétegek panel}: A rétegkezelés és tulajdonságok kezelésére szolgál
    \item \textbf{Munkaterület}: A képek szerkesztésének és megjelenítésének fő területe
\end{itemize}

\section{Képszerkesztés és Szűrők Alkalmazása}
\subsection{Alapvető Szerkesztések}
\begin{itemize}
    \item \textbf{Szürkeárnyalatos} (\texttt{Ctrl+G}): A kép szürkeárnyalatosra konvertálása
    \item \textbf{Invertálás} (\texttt{Ctrl+I}): A kép színeinek invertálása
    \item \textbf{Tükrözés} (\texttt{Ctrl+H}): A kép vízszintes tükrözése
    \item \textbf{Gamma korrekció}: A kép fényerejének beállítása
    \item \textbf{Logaritmikus transzformáció}: Dinamikatartomány tömörítése
\end{itemize}

\subsection{Speciális Szűrők}
\subsubsection{Elmosási szűrők}
Az elmosási szűrők a kép simítására szolgálnak:
\begin{itemize}
    \item \textbf{Box Filter}: Gyors átlagolás alapú elmosás
    \item \textbf{Gaussian Blur}: Sima elmosás testreszabható kernel mérettel (3x3, 5x5, 7x7)
\end{itemize}

\subsubsection{Élérzékelés}
Az élérzékelő szűrők a kép éleinek detektálására szolgálnak:
\begin{itemize}
    \item \textbf{Sobel éldetektor}: Sobel operátor alapú élérzékelés
    \item \textbf{Laplace éldetektor}: Laplace kernel alapú élérzékelés
\end{itemize}

\subsubsection{Harris Sarokérzékelés}
A Harris sarokdetektálás használata a következő lépésekből áll:
\begin{enumerate}
    \item A \textbf{Szűrők > Harris Sarokérzékelés} menüpont kiválasztása
    \item A paraméterek beállítása:
    \begin{itemize}
        \item \textbf{Érzékenység (k)}: A sarokválasz szabályozása
        \item \textbf{Küszöbérték}: A minimális sarokerősség (0-255 tartományban)
        \item \textbf{Gauss szigma}: Az előfeldolgozó elmosás erőssége
    \end{itemize}
    \item Az \textbf{Alkalmaz} gomb megnyomása a sarkok észleléséhez
\end{enumerate}

\section{Hisztogram Analízis}
\subsection{Hisztogram megjelenítése}
A hisztogram megjelenítése a következőképpen történik:
\begin{itemize}
    \item A \textbf{Hisztogram mutatása} menüpont kiválasztása
    \item A luminancia hisztogramjának megjelenítése
\end{itemize}

\subsection{Hisztogram kiegyenlítés}
A hisztogram kiegyenlítés végrehajtása:
\begin{enumerate}
    \item A \textbf{Hisztogram kiegyenlítés} menüpont kiválasztása
    \item A művelet automatikusan javítja a kép kontrasztját
    \item Az eredmény ellenőrizhető a hisztogram ablak újbóli megnyitásával
\end{enumerate}

\section{Rétegek Kezelése}
\subsection{Rétegek létrehozása és kezelése}
A rétegek kezelése a következő műveleteket tartalmazza:
\begin{itemize}
    \item \textbf{Új réteg}: Filter alkalmazásával egy új réteg jön létre a rétegek panelen
    \item \textbf{Destruktív szerkesztési mód}: Filter alkalmazásakor nem jön létre új réteg, hanem a meglévőben megy végbe a szerkesztés
    \item \textbf{Rétegek egyesítése}: A réteg kijelölése után az "Egyesítés" gomb használata
    \item \textbf{Duplikálás}: A réteg kijelölése után a "Duplikálás" gomb használata
    \item \textbf{Elrejtés}: A rétegen a checkboxra kattintás
    \item \textbf{Törlés}: A réteg kijelölése után a "Törlés" gomb használata
\end{itemize}

\subsection{Réteg tulajdonságok}
Az egyes rétegek tulajdonságainak kezelése:
\begin{itemize}
    \item \textbf{Láthatóság}: A checkboxra kattintással kapcsolható a réteg mutatása vagy elrejtése
\end{itemize}

\subsection{Rétegek egyesítése}
A rétegek kombinálásának lehetőségei:
\begin{itemize}
    \item \textbf{Rétegek egyesítése}: A kijelölt rétegek kombinálása
    \item \textbf{Összevonás}: Minden látható réteg egyetlen rétegbe fűzése
\end{itemize}

\section{Visszavonás és Ismétlés (Undo/Redo) műveletek}
A műveletek visszavonásának és ismétlésének lehetőségei:
\begin{itemize}
    \item \textbf{Visszavonás}: \texttt{Ctrl+Z} (maximum 50 műveletig)
    \item \textbf{Ismétlés}: \texttt{Ctrl+Y}
    \item \textbf{Támogatott műveletek}: Minden szűrő és rétegművelet visszavonható
\end{itemize}

\section{Billentyűparancsok}
\begin{table}[H]
\centering
\begin{tabular}{|l|l|}
\hline
\textbf{Billentyűkombináció} & \textbf{Művelet} \\
\hline
\texttt{Ctrl+N} & Új projekt létrehozása \\
\hline
\texttt{Ctrl+O} & Kép megnyitása \\
\hline
\texttt{Ctrl+S} & Kép mentése \\
\hline
\texttt{Ctrl+Shift+S} & Mentés másként \\
\hline
\texttt{Ctrl+Z} & Visszavonás \\
\hline
\texttt{Ctrl+Y} & Ismétlés \\
\hline
\texttt{Ctrl+G} & Szürkeárnyalatos konverzió \\
\hline
\texttt{Ctrl+I} & Színek invertálása \\
\hline
\texttt{Ctrl+H} & Vízszintes tükrözés \\
\hline
\texttt{Ctrl+F} & Illesztés a képernyőhöz \\
\hline
\texttt{Ctrl++} & Nagyítás \\
\hline
\texttt{Ctrl+-} & Kicsinyítés \\
\hline
\texttt{Ctrl+0} & 100\%-os nagyítás \\
\hline
\end{tabular}
\caption{Billentyűparancsok összefoglaló táblázata}
\end{table}

\section{Képek Mentése}
\subsection{Támogatott formátumok}
Az alkalmazás a következő képformátumok mentését támogatja:
\begin{itemize}
    \item \textbf{PNG} (Portable Network Graphics) - Veszteségmentes formátum
    \item \textbf{JPEG} (Joint Photographic Experts Group) - Tömörített formátum
\end{itemize}

\subsection{Mentési lehetőségek}
A mentés kétféleképpen történhet:
\begin{itemize}
    \item \textbf{Mentés} (\texttt{Ctrl+S}): Az aktuális fájl felülírása
    \item \textbf{Mentés másként} (\texttt{Ctrl+Shift+S}): Új fájl létrehozása
\end{itemize}

\chapter{Fejlesztői Dokumentáció}
\section{Architektúra és Fő Komponensek}
A Pixelium moduláris architektúrája két fő komponensből áll:

\subsection{Pixelium.Core}
A képfeldolgozási könyvtár, amely független a felhasználói felülettől:
\begin{itemize}
    \item Képfeldolgozási algoritmusok implementációja
    \item Rétegkezelés logikája
    \item Hisztogram számítások
\end{itemize}

\subsection{Pixelium.UI}
Avalonia UI alapú asztali alkalmazás:
\begin{itemize}
    \item Felhasználói felület komponensek
    \item MVVM minta implementációja
    \item Parancskezelés és visszavonás/ismétlés rendszer
\end{itemize}

\section{Kulcstechnológiák}
Az alkalmazás a következő technológiákra épül:
\begin{itemize}
    \item \textbf{.NET 9.0}: Modern, nagy teljesítményű alkalmazás keretrendszer
    \item \textbf{Avalonia UI}: Keretrendszertől független UI keretrendszer
    \item \textbf{SkiaSharp}: 2D grafikai könyvtár gyors rendereléshez
    \item \textbf{MVVM minta}: Tiszta választási rétegek megvalósítása
    \item \textbf{Parancs minta}: Visszavonás/ismétlés funkcionalitás implementációja
\end{itemize}

\clearpage
\section{Teljesítményoptimalizálások}
A Pixelium több teljesítményoptimalizálási technikát alkalmaz a gyors képfeldolgozás érdekében.

\subsection{Lookup Table (LUT) Gyorsítótárazás}
A lookup table-ök (keresési táblázatok) előre kiszámított transzformációs értékeket tárolnak, ami jelentősen felgyorsítja a pixelenkénti műveleteket.

\subsubsection{LUT szolgáltatás implementációja}
A \texttt{LookupTableService} osztály a következő funkciókat biztosítja:
\begin{itemize}
    \item \textbf{Thread-safe cache}: \texttt{ConcurrentDictionary<string, Lazy<byte[]>>} alapú gyorsítótár
    \item \textbf{Lazy inicializálás}: A LUT-ok csak az első használatkor kerülnek kiszámításra
    \item \textbf{Előre betöltött táblázatok}: A gyakran használt táblázatok (invert, grayscale, gamma) már inicializáláskor létrejönnek
    \item \textbf{256 elemű táblázatok}: Minden lehetséges bemeneti érték (0-255) számára előre kiszámított kimenet
\end{itemize}

\subsubsection{LUT alapú szűrők}
A következő szűrők használnak LUT-ot a gyors végrehajtáshoz:
\begin{itemize}
    \item \textbf{GrayscaleProcessor}: Három külön LUT-ot használ (R×0.299, G×0.587, B×0.114)
    \item \textbf{InvertProcessor}: Egyszerű 255-érték LUT
    \item \textbf{GammaProcessor}: Gamma érték alapján előre kiszámított táblázat
    \item \textbf{LogarithmicProcessor}: Logaritmikus transzformáció LUT-ja
\end{itemize}

\subsection{Párhuzamos Feldolgozás (Multithreading)}
A képfeldolgozó algoritmusok intenzíven használják a párhuzamos feldolgozást:

\subsubsection{Sor-alapú párhuzamosítás}
A képfeldolgozás soronként történik párhuzamosan a \texttt{Parallel.For} használatával:

\begin{lstlisting}[language={[Sharp]C}]
Parallel.For(0, bitmap.Height, y =>
{
    byte* rowPtr = ptr + (y * bitmap.Width * 4);
    for (int x = 0; x < bitmap.Width * 4; x += 4)
    {
        // Pixelenkenti feldolgozas
    }
});
\end{lstlisting}

\subsubsection{Thread-safe adatstruktúrák}
\begin{itemize}
    \item \textbf{ConcurrentDictionary}: A LUT gyorsítótár thread-safe tárolására
    \item \textbf{Lazy<T>}: Biztosítja, hogy a LUT-ok inicializálása csak egyszer történjen meg párhuzamos hozzáférés esetén is
    \item \textbf{Független sorok}: Minden képsor függetlenül dolgozható fel, nincs szükség szinkronizációra
\end{itemize}

\subsection{Unsafe Kód és Pointer Műveletek}
A kritikus teljesítményű részeken közvetlen memória-hozzáférés történik unsafe pointer műveletekkel:

\subsubsection{Előnyök}
\begin{itemize}
    \item \textbf{Gyorsabb hozzáférés}: Közvetlen memória manipuláció tömb indexelés helyett
    \item \textbf{Kevesebb bounds checking}: Nincs automatikus határellenőrzés
    \item \textbf{Cache-barát}: Lineáris memória hozzáférési minta
\end{itemize}

\subsubsection{Implementáció}
\begin{lstlisting}[language={[Sharp]C}]
var pixels = bitmap.GetPixels();
var ptr = (byte*)pixels.ToPointer();

// Kozvetlen memoria hozzaferes
byte blue = ptr[offset + 0];
byte green = ptr[offset + 1];
byte red = ptr[offset + 2];
byte alpha = ptr[offset + 3];
\end{lstlisting}

\clearpage
\section{Képmegjelenítés és Formátumok}

\subsection{Belső Formátum}
Minden belső képfeldolgozási művelet a \texttt{SKColorType.Bgra8888} formátumot használja:

\subsubsection{BGRA8888 formátum jellemzői}
\begin{itemize}
    \item \textbf{4 byte per pixel}: Kék, Zöld, Piros, Alfa csatornák
    \item \textbf{Csatorna sorrend}: B-G-R-A (nem R-G-B-A!)
    \item \textbf{8 bit per csatorna}: Minden csatorna 0-255 értéktartományban
    \item \textbf{Premultiplied alpha}: Az alfa csatorna előre megszorozva
\end{itemize}

\subsubsection{Memória elrendezés}
\begin{lstlisting}
Pixel offset: [0] [1] [2] [3] [4] [5] [6] [7] ...
Csatorna:      B   G   R   A   B   G   R   A  ...
              |-- Pixel 1 --| |-- Pixel 2 --|
\end{lstlisting}

\subsection{Be- és Kimeneti Formátumok}

\subsubsection{Bemeneti formátumok (Megnyitás)}
Az alkalmazás a következő formátumokat képes beolvasni:
\begin{itemize}
    \item \textbf{PNG (Portable Network Graphics)}
    \begin{itemize}
        \item Veszteségmentes tömörítés
        \item Támogatja az alfa csatornát (átlátszóság)
        \item Minden képtípushoz ajánlott
    \end{itemize}
    \item \textbf{JPEG (Joint Photographic Experts Group)}
    \begin{itemize}
        \item Veszteséges tömörítés
        \item Nincs alfa csatorna támogatás
        \item Fotók tárolására optimalizált
    \end{itemize}
    \item \textbf{BMP (Bitmap)}
    \begin{itemize}
        \item Tömörítetlen vagy RLE tömörített
        \item Nagy fájlméret
        \item Gyors betöltés
    \end{itemize}
\end{itemize}

\subsubsection{Kimeneti formátumok (Mentés)}
Az alkalmazás a következő formátumokba képes menteni:
\begin{itemize}
    \item \textbf{PNG}
    \begin{itemize}
        \item Veszteségmentes mentés
        \item Alfa csatorna megőrzése
        \item Minőségi paraméter: 100 (maximális tömörítési szint)
    \end{itemize}
    \item \textbf{JPEG}
    \begin{itemize}
        \item Veszteséges mentés
        \item Kisebb fájlméret
        \item Minőségi paraméter: 100 (maximális minőség)
        \item Az átlátszó területek fehér háttérrel kerülnek mentésre
    \end{itemize}
\end{itemize}

\subsubsection{Formátum konverzió}
Minden betöltött kép automatikusan \texttt{BGRA8888} formátumra konvertálódik:

\begin{lstlisting}[language={[Sharp]C}]
using var originalBitmap = SKBitmap.Decode(stream);
var bitmap = new SKBitmap(
    originalBitmap.Width,
    originalBitmap.Height,
    SKColorType.Bgra8888,
    SKAlphaType.Premul
);
using var canvas = new SKCanvas(bitmap);
canvas.DrawBitmap(originalBitmap, 0, 0);
\end{lstlisting}

\clearpage
\section{Képfeldolgozási Algoritmusok}

\subsection{Pixelenkénti Transzformációk}

\subsubsection{Szürkeárnyalatos Konverzió (Grayscale)}
A színes kép szürkeárnyalatos képpé alakítása a luminancia számításával:

\textbf{Képlet:}
\begin{equation}
Gray = 0.299 \times R + 0.587 \times G + 0.114 \times B
\end{equation}

\textbf{LUT implementáció:}
\begin{itemize}
    \item Három külön lookup table (R, G, B csatornákra)
    \item Előre kiszámított értékek: \texttt{redLut[i] = (byte)(i * 0.299)}
    \item Minden pixel: \texttt{gray = redLut[r] + greenLut[g] + blueLut[b]}
\end{itemize}

\subsubsection{Invertálás (Invert)}
A kép színeinek invertálása minden csatornán:

\textbf{Képlet:}
\begin{equation}
Output = 255 - Input
\end{equation}

\textbf{LUT implementáció:}
\begin{lstlisting}[language={[Sharp]C}]
invertLut[i] = (byte)(255 - i);  // i: 0-255
\end{lstlisting}

\subsubsection{Gamma Korrekció}
A kép fényerejének nem-lineáris beállítása:

\textbf{Képlet:}
\begin{equation}
Output = 255 \times \left(\frac{Input}{255}\right)^{1/\gamma}
\end{equation}

\textbf{Paraméterek:}
\begin{itemize}
    \item $\gamma < 1.0$: Világosítás
    \item $\gamma = 1.0$: Nincs változás
    \item $\gamma > 1.0$: Sötétítés
\end{itemize}

\subsubsection{Logaritmikus Transzformáció}
Dinamikatartomány tömörítése logaritmikus függvénnyel:

\textbf{Képlet:}
\begin{equation}
Output = c \times \log(1 + Input)
\end{equation}

ahol $c$ egy konstans a normalizáláshoz.

\clearpage
\subsection{Geometriai Transzformációk}

\subsubsection{Tükrözés (Flip)}
A kép tükrözése vízszintesen vagy függőlegesen:

\textbf{Vízszintes tükrözés:}
\begin{itemize}
    \item Minden sor pixeleinek megfordítása
    \item Idő komplexitás: $O(width \times height)$
    \item Párhuzamosan futtatható soronként
\end{itemize}

\textbf{Függőleges tükrözés:}
\begin{itemize}
    \item A sorok sorrendjének megfordítása
    \item Szintén párhuzamosítható
\end{itemize}

\clearpage
\subsection{Konvolúciós Szűrők}

\subsubsection{Konvolúció Általános Képlete}
A konvolúciós szűrők egy kernelt (maszkot) alkalmaznak a képre:

\begin{equation}
Output(x,y) = \sum_{i=-k}^{k} \sum_{j=-k}^{k} Input(x+i, y+j) \times Kernel(i,j)
\end{equation}

\subsubsection{Box Filter (Átlagoló Elmosás)}
Egyszerű átlagoló szűrő, minden kernelelem értéke azonos:

\textbf{3×3 kernel:}
\begin{equation}
\frac{1}{9}
\begin{bmatrix}
1 & 1 & 1 \\
1 & 1 & 1 \\
1 & 1 & 1
\end{bmatrix}
\end{equation}

\subsubsection{Gauss Elmosás (Gaussian Blur)}
Gauss-eloszlás alapú elmosás, simább eredményt ad:

\textbf{Gauss függvény:}
\begin{equation}
G(x,y) = \frac{1}{2\pi\sigma^2} e^{-\frac{x^2+y^2}{2\sigma^2}}
\end{equation}

\textbf{5×5 kernel példa ($\sigma=1.0$):}
\begin{equation}
\frac{1}{273}
\begin{bmatrix}
1 & 4 & 7 & 4 & 1 \\
4 & 16 & 26 & 16 & 4 \\
7 & 26 & 41 & 26 & 7 \\
4 & 16 & 26 & 16 & 4 \\
1 & 4 & 7 & 4 & 1
\end{bmatrix}
\end{equation}

\subsubsection{Sobel Élérzékelés}
A Sobel operátor két irányt (vízszintes és függőleges) vizsgál:

\textbf{Vízszintes kernel ($G_x$):}
\begin{equation}
\begin{bmatrix}
-1 & 0 & 1 \\
-2 & 0 & 2 \\
-1 & 0 & 1
\end{bmatrix}
\end{equation}

\textbf{Függőleges kernel ($G_y$):}
\begin{equation}
\begin{bmatrix}
-1 & -2 & -1 \\
0 & 0 & 0 \\
1 & 2 & 1
\end{bmatrix}
\end{equation}

\textbf{Eredmény számítás:}
\begin{equation}
G = \sqrt{G_x^2 + G_y^2}
\end{equation}

\subsubsection{Laplace Élérzékelés}
A Laplace operátor második deriváltakat használ:

\textbf{3×3 kernel:}
\begin{equation}
\begin{bmatrix}
0 & 1 & 0 \\
1 & -4 & 1 \\
0 & 1 & 0
\end{bmatrix}
\end{equation}

\clearpage
\subsection{Harris Sarokérzékelés}
A Harris sarokdetektor algoritmus a következő lépésekből áll:

\subsubsection{Algoritmus lépései}
\begin{enumerate}
    \item \textbf{Gradiens számítás}: Sobel operátorral $I_x$ és $I_y$ számítása
    \item \textbf{Gradiens négyzetek}: $I_x^2$, $I_y^2$, és $I_x \times I_y$ számítása
    \item \textbf{Gauss simítás}: A négyzetek simítása Gauss szűrővel
    \item \textbf{Sarokválasz számítás}:
    \begin{equation}
    R = det(M) - k \times trace(M)^2
    \end{equation}
    ahol $M$ a struktúra mátrix, $k$ tipikusan 0.04-0.06
    \item \textbf{Küszöbölés}: Csak a $R > threshold$ pontok kerülnek megjelölésre
    \item \textbf{Non-maximum suppression}: Helyi maximumok kiválasztása
\end{enumerate}

\subsubsection{Paraméterek}
\begin{itemize}
    \item \textbf{Threshold}: Minimális sarokerősség (0-255)
    \item \textbf{k}: Érzékenységi paraméter (0.04-0.06 ajánlott)
    \item \textbf{Sigma}: Gauss simítás erőssége
\end{itemize}

\clearpage
\subsection{Hisztogram Alapú Műveletek}

\subsubsection{Hisztogram Számítás}
A hisztogram egy 256 elemű tömb, amely az egyes intenzitásértékek gyakoriságát tárolja:

\begin{lstlisting}[language={[Sharp]C}]
var histogram = new int[256];
for (int i = 0; i < pixelCount; i++)
{
    byte intensity = pixels[i];
    histogram[intensity]++;
}
\end{lstlisting}

\subsubsection{Luminancia Hisztogram}
RGB képeknél a luminancia számítása:
\begin{equation}
L = 0.299 \times R + 0.587 \times G + 0.114 \times B
\end{equation}

\subsubsection{Hisztogram Kiegyenlítés}
A hisztogram kiegyenlítés növeli a kép kontrasztját:

\textbf{Algoritmus:}
\begin{enumerate}
    \item Hisztogram számítása
    \item Kumulatív eloszlásfüggvény (CDF) számítása:
    \begin{equation}
    CDF[i] = \sum_{j=0}^{i} histogram[j]
    \end{equation}
    \item Normalizálás és LUT létrehozása:
    \begin{equation}
    Output[i] = \frac{CDF[i] - CDF_{min}}{TotalPixels - CDF_{min}} \times 255
    \end{equation}
    \item LUT alkalmazása minden pixelre
\end{enumerate}

\clearpage
\section{Parancs Minta és Undo/Redo Funkció}
A Pixelium a Command Pattern-t alkalmazza a visszavonás/ismétlés funkcióhoz.

\subsection{IImageCommand Interfész}
Minden visszavonható művelet implementálja az \texttt{IImageCommand} interfészt:

\begin{lstlisting}[language={[Sharp]C}]
public interface IImageCommand
{
    void Execute();
    void Undo();
}
\end{lstlisting}

\subsection{CommandHistory}
A \texttt{CommandHistory} osztály kezeli a parancsok előzményeit:
\begin{itemize}
    \item \textbf{Maximum 50 parancs}: A legrégebbi parancsok automatikusan törlődnek
    \item \textbf{Stack alapú}: Végrehajtott és visszavont parancsok külön stacken
    \item \textbf{Snapshot}: Minden parancs elmenti a művelet előtti állapotot
\end{itemize}

\subsection{FilterCommand}
A szűrők alkalmazása előtt másolat készül a képről:
\begin{lstlisting}[language={[Sharp]C}]
public void Execute()
{
    _snapshot = _layer.Content.Copy();
    _processor.Process(_layer.Content);
}

public void Undo()
{
    var temp = _layer.Content;
    _layer.Content = _snapshot;
    _snapshot = temp;
}
\end{lstlisting}

\clearpage
\chapter{Képernyőképek}

\section{Alkalmazás Felület}

\begin{figure}[H]
\centering
\includegraphics[width=0.9\textwidth]{01_pixelium_home.png}
\caption{Pixelium főképernyő - Alapértelmezett nézetben a mintaképpel és a rétegkezelő panellel}
\label{fig:home}
\end{figure}

\clearpage
\section{Hisztogram Analízis}

\begin{figure}[H]
\centering
\includegraphics[width=0.9\textwidth]{02_pixelium_histogram.png}
\caption{Hisztogram megjelenítés - A luminancia eloszlás grafikus ábrázolása}
\label{fig:histogram}
\end{figure}

\clearpage
\section{Élérzékelés}

\begin{figure}[H]
\centering
\includegraphics[width=0.9\textwidth]{03_pixelium_sobel_edge_detection.png}
\caption{Sobel élérzékelés alkalmazása - Az élek kiemelése a képen}
\label{fig:sobel}
\end{figure}

\clearpage
\section{Réteges Szerkesztés}

\begin{figure}[H]
\centering
\includegraphics[width=0.9\textwidth]{04_pixelium_multiple_layers.png}
\caption{Több réteg kezelése - Rétegek panel különböző szűrőkkel}
\label{fig:layers}
\end{figure}

\clearpage
\section{Elmosási Szűrők}

\begin{figure}[H]
\centering
\includegraphics[width=0.9\textwidth]{05_pixelium_blur.png}
\caption{Gauss elmosás alkalmazása - A kép simítása Gauss szűrővel}
\label{fig:blur}
\end{figure}

\end{document}
